% --------------------------------------------------------------------------- %
% Poster for the ECCS 2011 Conference about Elementary Dynamic Networks.      %
% --------------------------------------------------------------------------- %
% Created with Brian Amberg's LaTeX Poster Template. Please refer for the     %
% attached README.md file for the details how to compile with `pdflatex`.     %
% --------------------------------------------------------------------------- %
% $LastChangedDate:: 2011-09-11 10:57:12 +0200 (V, 11 szept. 2011)          $ %
% $LastChangedRevision:: 128                                                $ %
% $LastChangedBy:: rlegendi                                                 $ %
% $Id:: poster.tex 128 2011-09-11 08:57:12Z rlegendi                        $ %
% --------------------------------------------------------------------------- %
\documentclass[a0paper,landscape]{baposter}

\usepackage{relsize}		% For \smaller
\usepackage{url}			% For \url
% \usepackage{epstopdf}	% Included EPS files automatically converted to PDF to include with pdflatex
\usepackage{tcolorbox}
%%% Global Settings %%%%%%%%%%%%%%%%%%%%%%%%%%%%%%%%%%%%%%%%%%%%%%%%%%%%%%%%%%%

\graphicspath{{pix/}}	% Root directory of the pictures 
% \tracingstats=2			% Enabled LaTeX logging with conditionals

%%% Color Definitions %%%%%%%%%%%%%%%%%%%%%%%%%%%%%%%%%%%%%%%%%%%%%%%%%%%%%%%%%

\definecolor{bordercol}{HTML}{440154FF}
\definecolor{headercol2}{HTML}{55C667FF}
\definecolor{headercol1}{HTML}{481567FF}
\definecolor{headerfontcol}{RGB}{256,256,256}
\definecolor{boxcolor}{RGB}{256,256,256}

%%%%%%%%%%%%%%%%%%%%%%%%%%%%%%%%%%%%%%%%%%%%%%%%%%%%%%%%%%%%%%%%%%%%%%%%%%%%%%%%
%%% Utility functions %%%%%%%%%%%%%%%%%%%%%%%%%%%%%%%%%%%%%%%%%%%%%%%%%%%%%%%%%%

%%% Save space in lists. Use this after the opening of the list %%%%%%%%%%%%%%%%
\newcommand{\compresslist}{
	\setlength{\itemsep}{1pt}
	\setlength{\parskip}{0pt}
	\setlength{\parsep}{0pt}
}

%%%%%%%%%%%%%%%%%%%%%%%%%%%%%%%%%%%%%%%%%%%%%%%%%%%%%%%%%%%%%%%%%%%%%%%%%%%%%%%
%%% Document Start %%%%%%%%%%%%%%%%%%%%%%%%%%%%%%%%%%%%%%%%%%%%%%%%%%%%%%%%%%%%
%%%%%%%%%%%%%%%%%%%%%%%%%%%%%%%%%%%%%%%%%%%%%%%%%%%%%%%%%%%%%%%%%%%%%%%%%%%%%%%

\begin{document}
\typeout{Poster rendering started}

%%% Setting Background Image %%%%%%%%%%%%%%%%%%%%%%%%%%%%%%%%%%%%%%%%%%%%%%%%%%
\background{
	\begin{tikzpicture}[remember picture,overlay]%
	\draw (current page.north west)+(-2em,2em) node[anchor=north west]
	{\includegraphics[height=1.1\textheight]{background}};
	\end{tikzpicture}
}

%%% General Poster Settings %%%%%%%%%%%%%%%%%%%%%%%%%%%%%%%%%%%%%%%%%%%%%%%%%%%
%%%%%% Eye Catcher, Title, Authors and University Images %%%%%%%%%%%%%%%%%%%%%%
\begin{poster}{
	grid=false,
	% Option is left on true though the eyecatcher is not used. The reason is
	% that we have a bit nicer looking title and author formatting in the headercol
	% this way
	%eyecatcher=false, 
	borderColor=bordercol,
	headerColorOne=headercol1,
	headerColorTwo=headercol2,
	headerFontColor=headerfontcol,
	% Only simple background color used, no shading, so boxColorTwo isn't necessary
	boxColorOne=white, %boxcolor,
	bgColorOne=white,
	headershape=roundedright,
	headerfont=\Large\sf\bf,
	textborder=rectangle,
	background=plain,
	headerborder=open,
  boxshade=plain
}
%%% Eye Cacther %%%%%%%%%%%%%%%%%%%%%%%%%%%%%%%%%%%%%%%%%%%%%%%%%%%%%%%%%%%%%%%
{
	Eye Catcher, empty if option eyecatcher=false - unused
}
%%% Title %%%%%%%%%%%%%%%%%%%%%%%%%%%%%%%%%%%%%%%%%%%%%%%%%%%%%%%%%%%%%%%%%%%%%
{\sf\bf
	Can Massive Rocky Exoplanets Have a Solid Inner Core? 
}
%%% Authors %%%%%%%%%%%%%%%%%%%%%%%%%%%%%%%%%%%%%%%%%%%%%%%%%%%%%%%%%%%%%%%%%%%
{
	\vspace{1em} M. Lasbleis$^{1,2}$, L. Noack$^3$\\
	{\smaller $^1$ Laboratoire de Plan\'etologie et G\'eodynamique de Nantes, Univ. de Nantes, CNRS, France \\$^2$ ELSI, Tokyo Inst. of Tech., Japan $^3$  Free University, Berlin, Germany}
}
%%% Logo %%%%%%%%%%%%%%%%%%%%%%%%%%%%%%%%%%%%%%%%%%%%%%%%%%%%%%%%%%%%%%%%%%%%%%
{
% The logos are compressed a bit into a simple box to make them smaller on the result
% (Wasn't able to find any bigger of them.)

		\begin{minipage}{0.20\textwidth}
		\hfill
		\includegraphics[height=0.18\textwidth]{logo.pdf}
		\hfill
         \includegraphics[height=0.18\textwidth]{GeoPlaNet.pdf}
         \hfill
         \includegraphics[height=0.18\textwidth]{logo_Nantes.pdf}
         \hfill
         
         \hfill
         \includegraphics[height=0.2\textwidth]{lpg.pdf}
         \hfill
         \includegraphics[height=0.2\textwidth]{FreeUni.pdf}
         \hfill
         \includegraphics[height=0.2\textwidth]{CNRS.pdf}
         \hfill
%			\includegraphics[width=10em,height=4em]{colbud_logo}

		\end{minipage}
	}



\headerbox{}%
{name=foottext, column=0, span=4,above=bottom,%
 textborder=none,headerborder=none,boxheaderheight=0pt, boxColorOne=white}{
         \hfill
         \begin{minipage}{0.3\textwidth}
                 \begin{center}
                         marine.lasbleis@univ-nantes.fr\\
                         http://marinelasbleis.github.io/
                 \end{center}
         \end{minipage}
         \hfill
         \begin{minipage}{0.4\textwidth}
         This project started during an ELSI-Origins-Network workshop.  ML is currently funded under the GeoPlanet consortium, funded by the Region Loire. 
         \end{minipage}
         }


\headerbox{Scientific question}{name=question,column=0,row=0, span=1}{
\textbf{What can we say about the thermal structure of the core of an exoplanet? }

\vspace{1em}

Here, we investigate the structure of a rocky planet shortly after formation. Internal structures of planets are calculated based on previous work by \cite{Noack2017}, providing pressure, temperature and material properties profiles that are needed to understand the thermal evolution of mantle and core of these planets. We explore planetary masses from 0.8 to 2 Earth masses, considering variations of compositions leading to different core sizes. 

\vspace{1em}

Following \cite{Stixrude2014}, we consider the thermal profile of the core after the full crystallization of the silicates at the Core Mantle Boundary (CMB). Anchoring the profile at the melting temperature of the mantle at the top of the outer core, we may thus overestimate the possible temperatures in the core. 
\begin{itemize}\compresslist
    \item Planet masses: 0.8 to 2 Earth mass
    \item Iron fraction: 15\% to 75\%
    \item Mantle iron number: 0 to 20
\end{itemize}
(most results are shown for iron number of 0)

\vspace{1em}
\includegraphics[width=\textwidth]{singleplanet.pdf}
}

\headerbox{Interior structures}{name=struct,column=1, row=0,span=2}{
Interior structure model derived from \cite{Noack2017}. The total mass of the planet and its iron composition are fixed, the model determines self-consistently the structure and radius, using temperature profiles as described below. 
\begin{center}
    \includegraphics[width=0.9\textwidth]{radius_poster}

\includegraphics[width=0.99\textwidth]{profiles_T_P_rho_g.pdf}
\end{center}
}

\headerbox{Temperature profile}{name=temperature,column=1, below=struct, span=2,above=foottext}{


\begin{minipage}{0.60\textwidth}
\includegraphics[width=\textwidth]{Tc_M_Fe.pdf}

\includegraphics[width=\textwidth]{temperature_profiles.pdf}

\end{minipage}
\hfill
\begin{minipage}{0.4\textwidth}
Following \cite{Stixrude2014}, the temperatures are anchored at the top of the mantle and core by the melting temperature of, respectively, the upper mantle and the bottom mantle, assuming vigorous convection in both layers. The melting temperature of iron (and iron alloys) is used to determine the inner core.

\vspace{1em}

\hline

\vspace{1em}

\textbf{Melting temperatures ($P>17$GPa):}

\begin{equation*}
    T_{\rm mantle} = 5400 * \left (\frac{p[\rm {GPa}]}{140}\right )^{0.48} \frac{1}{1 - \ln(1-\#Fe_{\rm M}) }
\end{equation*}

\begin{equation*}
    T_{\rm core} = 6500 * \left (\frac{p[GPa]}{340}\right)^{0.515} \frac{1}{1 - \ln(1-X_{\rm S}) }
\end{equation*}



\hline

\vspace{1em}

For iron alloy, the liquidus temperature depends on the light element and is not well known. 
\end{minipage}
}

\headerbox{Inner core size distribution}
{name=IC,span=1,column=3, row=0}{
\includegraphics[width = \textwidth]{Radius_planet_core_inner_core.pdf}
\includegraphics[width=\textwidth]{Delta_T.pdf}

}





\headerbox{Conclusions and perspectives}{name=ccl,span=1,column=3,below=IC,above=foottext}{
\begin{itemize}\compresslist
    \item Inner core crystallization during accretion is likely to be a common process for planets with large cores. 
    \item Presence and size of early IC do not strongly vary with planet mass, but with core fraction and core impurities.
\end{itemize}

Such early IC are likely to change the usual view on the thermal and magnetic evolution of planetary cores (see \cite{Driscoll2011}). This work provides a first step for structure after mantle freezing. Stay tuned for more time evolution. 

\includegraphics[width=\textwidth]{time_evolution_onlyCFeM0_mass_1&2.pdf}
}



\headerbox{References}{name=references,column=0,below=question,above=foottext}{
\smaller													% Make the whole text smaller
\vspace{-0.4em} 										% Save some space at the beginning
\bibliographystyle{apalike}							% Use plain style
\renewcommand{\section}[2]{\vskip 0.05em}		% Omit "References" title
\begin{thebibliography}{1}							% Simple bibliography with widest label of 1
\itemsep=-0.01em										% Save space between the separation
\setlength{\baselineskip}{0.4em}					% Save space with longer lines
% \bibitem{prevWork1} Laszlo Gulyas, Richard Legendi: \emph{Effects of Sample Duration on Network Statistics in Elementary Models of Dynamic Networks}, International Conference on Computational Science, Singapore (2011) 
    \bibitem{Stixrude2014} Stixrude, L. \emph{Melting in super-earths.} Philosophical Transactions of the Royal Society of London A: Mathematical, Physical and Engineering Sciences 372.2014 (2014): 20130076.
    \bibitem{Noack2017} Noack, Lena, Attilio Rivoldini, and Tim Van Hoolst. \emph{Volcanism and outgassing of stagnant-lid planets: Implications for the habitable zone.} Physics of the Earth and Planetary Interiors 269 (2017): 40-57.
    \bibitem{Driscoll2011} Driscoll, Peter, and Peter Olson. \emph{Optimal dynamos in the cores of terrestrial exoplanets: Magnetic field generation and detectability.} Icarus 213.1 (2011): 12-23.
\end{thebibliography}


}

\end{poster}
\end{document}
